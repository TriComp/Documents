
\documentclass{article}

\usepackage[top=3cm, bottom=3cm, left=3cm, right=3cm]{geometry}

\usepackage[french]{babel}
\usepackage[utf8]{inputenc}
\usepackage[T1]{fontenc}

\usepackage{graphicx}
\usepackage{caption}

\usepackage{tikz}
\usetikzlibrary{positioning,shapes,arrows,automata}
\tikzset{>=stealth'}

\begin{document}
 
     \begin{figure}[!h]
    \centering

    \begin{tikzpicture}[->,>=stealth',shorten >=1pt,auto,node distance= 2cm,
                    semithick]
      \tikzstyle{every state}=[draw=black,text=black,shape=rectangle]

      \node[state]     (niv1)   []                     {Spécifications par l'utilisateur};
      \node[state]     (niv2)	[] [below of=niv1]     {Langage descriptif};
      \node[state]     (niv3) 	[] [below of=niv2]     {Représentation intermerdiaire (graphe)};
      \node[state]     (niv4)   [] [below of=niv3]     {Langage bas niveau};
      \node[state]     (niv5)   [] [below of=niv4]     {Instructions pour l'utilisateur};


      \path (niv1)   edge               node {GUI} 	   (niv2)
	    (niv2)   edge 	 	node {Compilateur} (niv3)
	    (niv3)   edge               node {Compilateur} (niv4)
	    (niv4)   edge               node {GUI}	   (niv5)
      ;

    \end{tikzpicture}

    \caption{Cinq niveau de langages}
    \label{nvx-lg}
    \end{figure}
 
\end{document}
