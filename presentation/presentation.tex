\documentclass{beamer}
\usetheme{Boadilla}

\usepackage[utf8]{inputenc}
\usepackage[french]{babel} %pour la typographie française.
\usepackage[T1]{fontenc} %pour la césure des mots.
\usepackage{amssymb} %pour les symboles mathématiques et lettres grecques
\usepackage{amsmath} %pour les formules mathématiques avancées
\usepackage{listings} %lstlisting
\usepackage{xcolor} % pour mettre du texte en couleur

\beamertemplatenavigationsymbolsempty % Flag nazi pour supprimer la ligne d'option dans les slides

\title{Projet TriComp : Présentation}
\author{Equipe TriComp}
\date{Jeudi 18 Décembre 2014}

\begin{document}

\frame{
   \begin{center}
        \includegraphics[height=0.2\textheight]{../img/ball_of_wool.png}
        \hspace*{4cm}
        \includegraphics[height=0.2\textheight]{../img/ens.jpg}
   \end{center}
   \titlepage
}

\frame{
	\frametitle{Plan} 
		\tableofcontents
}

% Proposition de plan :
% Introduction : présentation de l'équipe, 
% TriComp : Origine de l'idée, Compilateur ? Tricot ?
% Travail en amont : Problèmes à résoudre (allocation aiguilles, qqchose de général) -> "langage haut niveau"
% Coté logiciel : interface
% Difficultés rencontrées
% Améliorations possibles (complexifier fonctionnalites, reprendre les objectifs non traités du midterm)

\section{Introduction}

\subsection{Présentation générale}

\frame{
	\frametitle{L'équipe TriComp}
		\begin{center}
			{\large
			{William \textsc{Aufort} \hspace{4cm} Julien \textsc{Bensmail} \\}
			\vspace{1cm}
			{Agathe \textsc{Herrou}  \hspace{4cm} Romain \textsc{Labolle} \\}
			\vspace{1cm}
			{Frédéric \textsc{Lang} \hspace{4cm} Maxime \textsc{Lesourd} \\}
			\vspace{1cm}       
			{Laureline \textsc{Pinault} \hspace{3.7cm} Léo \textsc{Stéfanesco} \\}}
		\end{center}
}

\frame{
	\frametitle{TriComp ?}
		\begin{center}
			\begin{tabular}{ccc}
		        \onslide<2-> \includegraphics[height=0.35\textheight]{../img/tricot.png} &
				\onslide<3-> \includegraphics[height=0.35\textheight]{../img/plus.jpg} &
				\includegraphics[height=0.35\textheight]{../img/gcc.png}
			\end{tabular}
		\end{center}
}

\frame{
	\frametitle{Présentation générale}
		\begin{itemize}
			\onslide<1->\item Un compilateur pour tricot
			\onslide<2-> \item D'un patron de tricot ... \onslide<3-> vers une liste d'instructions
			\onslide<4-> \item Le tout utilisable via une interface graphique
		\end{itemize}
}

\subsection{Quelques notions de tricot}

\frame{
	\frametitle{Quelques notions de tricot...}
        \begin{columns}   
          \column{0.6\textwidth}
          \includegraphics[scale=0.3]{Knit-schematic.png}
          \column{0.3\textwidth}
          \begin{itemize}
            \pause % à savoir : plus simple d'utilisation que \onslide
          \item[] \textcolor{green}{maille}
            \pause
          \item[] \textcolor{red}{rang}
          \end{itemize}
        \end{columns}
}

% TODO : Maille à l'endroit, à l'envers ---> Un panel de ce qu'on peut faire avec ? 

\frame{
	\frametitle{Les instructions de tricot}
		\begin{itemize}
			\onslide<2-> \item Les instructions à suivre par le tricoteur comportent des informations sur :
			\begin{itemize}
				\onslide<3-> \item le type de point à tricoter
				\onslide<4-> \item le nombre de points identiques
				\onslide<5-> \item le nombre de rangs à reproduire
			\end{itemize}
			% TODO : une image avec des instructions ? 
		\end{itemize}
}

\frame{
	\frametitle{Quelques notions de tricot...}
        \begin{columns}   
          \column{0.6\textwidth}
          \includegraphics[scale=0.3]{Knit-schematic.png}
          \column{0.4\textwidth}
          \begin{itemize}
          \item[] \textcolor{green}{maille à l'endroit}
          \item[]
          \end{itemize}
        \end{columns}
}			
\frame{
	\frametitle{Quelques notions de tricot...}
        \begin{columns}   
          \column{0.6\textwidth}
          \includegraphics[scale=0.3]{Knit-schematic2.png}
          \column{0.4\textwidth}
          \begin{itemize}
          \item[] \textcolor{green}{maille à l'endroit}
          \item[] \textcolor{blue}{maille à l'envers}
          \end{itemize}
        \end{columns}
}			

\subsection{Motivations}

\frame{
	\frametitle{Motivations}
		\begin{itemize}
			\onslide<2-> \item Le tricot n'est pas très présent dans le milieu informatique...
			\onslide<3-> \item Les logiciels de tricot actuels ne proposent souvent que de l'édition
			\onslide<4-> \item De plus, ces logiciels utilisent une représentation très locale des points de tricot.
			% TODO : image 
						% Deux raisons d'automatiser le calcul des instructions			
			\onslide<5-> \item Des instructions longues, compliquées à établir, et qui ne tiennent pas compte de la forme du tricot dans les ouvrages.
			\onslide<6-> \item Un projet original !
			% TODO : autres ?
		\end{itemize}
}

\frame{
	\frametitle{Plan} 
		\tableofcontents
}

\section{Notre travail}

\subsection{Conception d'un langage descriptif}	

\frame{
	\frametitle{En amont... description d'un tricot}
		\begin{itemize}
			\onslide<2-> \item Un schéma de description de nimporte quel tricot
			\onslide<3-> \item Plusieurs critères :
			\begin{itemize}
				\onslide<4-> \item gérer le plus de tricots possibles
				\onslide<5-> \item facilité pour générer les instructions
				\onslide<6-> \item assez "haut niveau" % i.e : suffisamment précis pour avoir du détail, mais conserver un aspect global, contrairement
										  % à ce qui se fait d'habitude dans les logiciels de tricots (points sur une grille)
				\onslide<7-> \item facile de compréhension et d'utilisation (notamment dans le code...)
			\end{itemize}
			\onslide<8-> \item Nous avons donc conçu le langage TriLang % TODO : meilleur nom ?
		\end{itemize}
}

\frame{
	\frametitle{Le langage}
		\begin{itemize}
			\onslide<1-> \item La composante essentielle : le trapèze \\
			\onslide<2-> \begin{center}
				\includegraphics[height=0.4\textheight]{../img/trapeze.jpg}
			\end{center}
			\onslide<3-> \item Chaque trapèze pointe vers l'entité suivante $\rightarrow$ parcours facile du tricot
			\onslide<4-> \item Pour relier/séparer des pièces : \texttt{link} et \texttt{split}
			\onslide<5-> \item Un exemple
		\end{itemize}
}

\frame{
	 \frametitle{Un exemple}
}

\subsection{Conception d'un compilateur}

\frame{
	\frametitle{Le compilateur}
		\begin{itemize}
			\item Pouvoir passer d'un tricot à la liste des instructions pour le réaliser
			\item Mais aussi des étapes intermédiaires :
			\begin{itemize}
				\item Vérifications
			    \item Allocation d'aiguilles ? % TODO est ce que ca sera fait ?
			\end{itemize}
		\end{itemize}
}

% TODO : Mettre des résultats théoriques ici !

\subsection{Elaboration d'une interface}

\frame{
	\frametitle{L'interface graphique}
		\begin{itemize}
			\onslide<2-> \item Simple d'utilisation
			\onslide<3-> \item Outils d'éditions
			\onslide<4-> \item Appel au compilateur
		\end{itemize}
}

\frame{
	\begin{center}
		\Huge{Démo !}
	\end{center}
}


\frame{
	\frametitle{Plan} 
		\tableofcontents
}

\section{Difficultés rencontrées \& améliorations possibles}

\subsection{Difficultés}

\frame{
	\frametitle{Difficultés rencontrées}
		\begin{itemize}
			\onslide<2-> \item Première étape de conception du langage sous-estimée
			\onslide<3-> \item Mise en route difficile
			% TODO : Je vois pas trop comment on peut tourner cela.
                \end{itemize}
}

\frame{
	\frametitle{Mais quelques bonnes surprises}
		\begin{itemize}
			\onslide<2-> \item L'intégration des différents modules plus simple que prévue
			\onslide<3-> \item D'importantes simplifications par rapport aux modules % i.e moins de niveaux de langages (inutiles, regroupés),...
		\end{itemize}
}

\subsection{Améliorations}

\frame{
	\frametitle{Améliorations en perspective...}
	% i.e les idées intéressantes que nous n'avons pas eu le temps de traiter.
		\begin{itemize}
			\onslide<2-> \item Côté édition, plus de liberté dans la personnalisation :
			\begin{itemize}
				\onslide<3-> \item Des points définis par l'utilisateur
				\onslide<4-> \item Pouvoir ajouter des tresses $\longrightarrow$ séparations de trapèzes ?
				\onslide<5-> \item Changer la taille des pièces, ...
			\end{itemize}
			\onslide<6-> \item Côté compilateur :
			\begin{itemize}
				\onslide<7-> \item gérer les coutures (à effectuer après le tricot)
			\end{itemize}
		\end{itemize}
}

\frame{
	\begin{center}
		\Huge{Des questions ?}
	\end{center}
}	

\end{document}

