
\documentclass{article}


\usepackage[french]{babel}
\usepackage[T1]{fontenc}

\title{Proposition de projet : TriComp}

\author{}

\date{29 Septembre 2014}


\begin{document}

\maketitle
\pagebreak

% Esquisse de plan ( proposee ) :
%
% I] Introduction générale du projet
% II] Motivations et objectifs 
%     a) Motivations :
%        - un sujet surprenant et interessant, en soi et dans son interaction avec l'informatique.
%        - produits ou services concurrents/existants (services/+/-) --> Necessite de certaines fonctionnalites nouvelles.
%     b) objectifs : formalisation, compilation (modele -> instructions/manuel), realisation logicielle,...
%     c) public vise et retombees attendues 
% III] Organisation du projet 
%     a) Description de l'equipe (nombre/nom/prenom des collaborateurs, affinités/compétences variées...) 
%     b) Differentes phases avec calendrier / organisation par versions successives (plus simple de proceder ainsi du fait
%     de la complexite croissante des points. en plus cela permet d'avoir au moins une version "fonctionnelle"
%     "assez rapidement"/ 
%     c) calendrier + precis.


\section{Introduction}

Ce document présente la proposition de notre projet intitulé TriComp.
% Résumé du projet.
% Présentation rapide de l'équipe et du chef d'équipe ?

\section{Motivations et objectifs}

\subsection{Motivations}

\subsection{Objectifs}

\subsubsection{Formalisation d'un langage}

\subsubsection{Compilation}

\subsubsection{Réalisation logicielle : interface pour tricoteur}

\subsection{Public visé et retombées attendues}

\section{Organisation du projet}

\subsection{L'équipe}

\subsection{Les différentes phases et le mode d'organisation}

\subsection{Calendrier prévisionnel}


\end{document}
