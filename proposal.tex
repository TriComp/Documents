
\documentclass{article}

\usepackage[utf8]{inputenc}
\usepackage[french]{babel}
\usepackage[utf8]{inputenc}
\usepackage[T1]{fontenc}

\title{Proposition de projet : TriComp}

\author{}

\date{29 Septembre 2014}


\begin{document}

\maketitle
\pagebreak

% Esquisse de plan ( proposee ) :
%
% I] Introduction générale du projet
% II] Motivations et objectifs 
%     a) objectifs : formalisation, compilation (modele -> instructions/manuel), realisation logicielle,...
%     b) Motivations :
%        - un sujet surprenant et interessant, en soi et dans son interaction avec l'informatique.
%        - produits ou services concurrents/existants (services/+/-) --> Necessite de certaines fonctionnalites nouvelles.
%     c) public vise et retombees attendues 
% III] Organisation du projet 
%     a) Description de l'equipe (nombre/nom/prenom des collaborateurs, affinités/compétences variées...) 
%     b) Differentes phases avec calendrier / organisation par versions successives (plus simple de proceder ainsi du fait
%     de la complexite croissante des points. en plus cela permet d'avoir au moins une version "fonctionnelle"
%     "assez rapidement"/ 
%     c) calendrier + precis.

% Esquisse de plan ( proposee ) bis :
%
% I] Introduction et objectifs
% II) Produits concurrents
% II] Cahier des charges et applications 
%     a) Fonctionnalités :
%     b) Aspects techniques : les différents modules (gui, compilation, lg haut niveau,...) + diagrammes
%     c) public vise et retombees attendues 
% III] Organisation du projet 
%     a) Description de l'equipe : (nombre/nom/prenom des collaborateurs, affinités/compétences variées...) + moyan de communication,.. au sein de l'équipe (git, pad, réunions, ...)
%     b) Les différentes phases : Modèle par couches (versions successives)
%     c) Workpackages
%     d) Calendrier des taches



\section{Introduction et objectifs}

Ce document présente la proposition de notre projet intitulé TriComp.
% Résumé du projet.
% Présentation rapide de l'équipe et du chef d'équipe ?


\section{Produits concurrents}

\section{Objectifs et motivations}

\subsection{Fonctionnalités}

\subsection{Aspects techniques}

Les objectifs de ce projet sont assez diversifiés. 
Bien qu'il s'agisse d'un projet logiciel, les objectifs ne sont pas tous orientés directement vers le logiciel.

\subsubsection{Formalisation d'un langage}

La description d'un langage pour décrire les tricots ainsi que sa formalisation est un objectif primordiale.
Une telle description se fera par une formation continue en tricot, qui permettra d'ajouter de nouveaux points ( i.e de nouvelles
possibilités pour le tricot ) tout au long du projet, mais également d'avoir une vue plus générale sur la manière de concevoir un tel
langage. La formalisation du langage résultant sera également un objectif clé de ce projet.

\subsubsection{Compilation}

C'est l'intitulé du projet : construire un compilateur pour tricot. Le compilateur aura pour rôle de transformer le tricot que 
l'utilisateur souhaite créer en une succession d'instructions que celui-ci pourra réaliser. Le compilateur devra également être capable de 
détecter des impossibilités au niveau de la conception du modèle.

La conception du compilateur présuppose également d'avoir défini ( tout comme nous aurons défini le langage de la section précédente ) un 
langage pour les instructions qui seront générées ( un peu comme dans un manuel de tricot ). 
Une fois ce langage défini, nous pourrons nous concentrer sur la phase de conception du compilateur. 

\subsubsection{Réalisation logicielle : interface pour tricoteur}

Notre réalisation logicielle a pour but de mettre le compilateur précédent au service des utilisateurs. 
L'objectif ici est de concevoir un logiciel fournissant à ses utilisateurs essentiellement deux fonctionalités :
\begin{itemize} 
  \item \textbf{La conception d'un tricot via une interface graphique} Notre objectif ici est de mettre en place pour l'utilisateur des
  outils élémentaires pour qu'il puisse entrer son projet de tricot : choix d'un modèle ( pull, écharpe...), choix des points utilisés 
  ( jersey, ...), de la couleur.
  \item \textbf{Le suivi personalisé} Une fois que l'utilisateur aura saisi son tricot, le logiciel appelera le compilateur afin de générer 
la liste des instructions qui devront être suivies pour réaliser le tricot. Ces instructions permettront un suivi du tricoteur par le 
logiciel : le tricoteur pourra à travers le logiciel suivre les différentes étapes, avec notamment des illustrations de la tâche à 
accomplir et l'observation du résultat à obtenir à chaque étape.
\end{itemize}

\subsection{Motivations}

\subsection{Public visé et retombées attendues}

\section{Organisation et planification}

\subsection{L'équipe}

\subsection{Les différentes phases}

\subsection{Workpackages}

\subsection{Calendrier des tâches}


\end{document}
