\documentclass{article}

\usepackage[top=3cm, bottom=3cm, left=3cm, right=3cm]{geometry}

\usepackage[utf8]{inputenc}
\usepackage[T1]{fontenc}
\usepackage[frenchb]{babel}

\usepackage[a4paper,colorlinks,linkcolor=darkgray,citecolor=red,urlcolor=blue]{hyperref}
\usepackage{pdfpages}
\usepackage{graphicx}
\usepackage{caption}
\usepackage{amsthm}
\usepackage{listings}
\usepackage{tikz}
\usepackage{algorithm}
\usepackage{amsmath}
\usepackage[noend]{algpseudocode}
\usepackage{amsfonts}
\usepackage{amssymb}

\usetikzlibrary{calc, trees, positioning, arrows, shapes,
  shapes.multipart, shadows, matrix, decorations.pathreplacing,
  decorations.pathmorphing, automata, decorations.markings}

\tikzstyle{vecArrow} = [thick, decoration={markings,mark=at position
   1 with {\arrow[semithick]{open triangle 60}}},
   double distance=1.4pt, shorten >= 5.5pt,
   preaction = {decorate},
   postaction = {draw,line width=1.4pt, white,shorten >= 4.5pt}]
\tikzstyle{innerWhite} = [semithick, white,line width=1.4pt, shorten >= 4.5pt]
  
\definecolor{vert}{RGB}{0,153,0}
\definecolor{rouge}{RGB}{153,0,0}
\definecolor{bleu}{RGB}{0,0,153}

\newtheorem{ex}{Exemple}

\title{Rapport final : TriComp}

\author{}

\date{Vendredi 19 Décembre 2014}

\begin{document}

\lstset{
     literate=%
         {é}{{\'e}}1
}

\makeatletter % Pour utiliser le "at" comme une commande interne.
  \begin{titlepage}
    \begin{center}
       {\LARGE \@title} \\
       \vspace{2cm} {\large \@date}
       \vspace{3cm}
    \end{center}
       {\large {William \textsc{Aufort} \hfill Julien
           \textsc{Bensmail} \\}
    \vspace{1cm} {\hfill coordinateur \\} {Agathe \textsc{Herrou}
      \hfill Romain \textsc{Labolle} \\}
       \vspace{1cm} {chef de projet \\}
       \vspace{1.5cm} {Frédéric \textsc{Lang} \hfill Maxime
         \textsc{Lesourd} \\} {Laureline \textsc{Pinault} \hfill Léo
         \textsc{Stéfanesco} \\}}
       \vspace{2.5cm}
    \begin{abstract}
Ce document présente le rapport final de notre projet TriComp. Ce projet consiste en le développement d'un logiciel, TriComp, 
qui permettrait à l'utilisateur d'obtenir les instructions à effectuer pour tricoter l'ouvrage de son souhait, à partir de la 
description globale de celui-ci.

Nous y détaillons tout le travail que nous avons effectué, les considérations théoriques qui y ont mené, ainsi que les améliorations 
que nous avons envisagées.
    \end{abstract}
  \end{titlepage}
\makeatother

\newpage

\tableofcontents

\newpage

\section*{Introduction}

Le tricot est un art que l'on imagine souvent réservé aux personnes
âgées, mais qui pourtant s'adresse à un public bien plus diversifié. 
Un intérêt 

Les magazines de tricot présentent souvent des modèles à réaliser
soi-même, combinant différents points. Il est cependant vite
fastidieux d'écrire la suite des instructions à suivre pour réaliser
une pièce. Notre but ici était donc de réaliser un logiciel permettant
d'une part de décrire des pièces de tricot de manière
\emph{user-friendly}, et, à partir de cette description, de générer
une suite d'instructions à destination du tricoteur permettant de
réaliser la pièce en question.

Cette opération de traduction d'un langage de «\,haut niveau\,»
(l'ensemble des pièces du tricot) vers un langage de «\,bas niveau\,»
(les instructions) rappelait fortement le principe d'un compilateur,
c'est pourquoi nous avons d'abord considéré notre projet comme un
«\,compilateur de tricot\,». À l'instar de la compilation d'un langage
de programmation informatique, cette compilation de tricot était
susceptible de mener à des problèmes théoriques intéressants, à mesure
que l'on enrichirait le langage.

\subsection*{Notions de tricot} 

Le vocabulaire du tricot peut sembler un peu obscur aux non-initiés.
Voici donc quelques définitions qui seront utiles pour la compréhension
de la suite du rapport :

\begin{itemize}
% + rang
\item[$\bullet$] une \emph{maille} est l'unité de base d'un tricot. Il s'agit de
  la boucle qui constitue l'étoffe en étant reliée à ses voisines,
  horizontalement car partageant la même portion de fil, verticalement
  car ces boucles sont imbriquées les unes dans les autres. Elle peut
  être à l'endroit ou à l'envers, selon la manière dont la boucle est
  réalisée et imbriquée dans la boucle précédente. La figure \ref{maille} 
  montre comment sont composées des mailles et les rangs.

\begin{figure}[!ht]
  \centering \includegraphics[scale=0.25]{../presentation/Knit-schematic2.png}
  \caption{Un schéma sur une partie de tricot, où l'on peut observer des rangs 
  (en rouge), ainsi que des mailles endroit (en vert) et envers (en bleu)}
  \label{maille}
\end{figure}

\item[$\bullet$] un \emph{point} est une manière de combiner les différentes
  opérations réalisables sur les mailles (les tricoter à l'endroit, à
  l'envers, en tricoter plusieurs ensemble, croiser plusieurs mailles\dots).
  Ceci montre que l'on peut obtenir un nombre de motifs possibles 
  extrêmement important.
\item[$\bullet$] l'\emph{ouvrage} désigne la pièce de tricot tout entière, par
  exemple l'écharpe ou le pull.
\item[$\bullet$] une \emph{diminution} est la suppression d'une maille dans le
  cours de l'ouvrage ; une technique courante consiste à tricoter
  ensemble deux mailles. Symétriquement, une \emph{augmentation} est
  une maille ajoutée dans le cours de l'ouvrage ; une technique
  courante consiste à enrouler le fil sur l'aiguille pour former une
  nouvelle maille (on parle alors de \emph{jeté}).
\end{itemize}

Le tricot est réalisé rang par rang, à l'aide d'aiguilles, qui servent 
d'une part à maintenir les boucles libres (pour éviter qu'elles ne se 
défassent et ne libèrent les boucles du rang précédent qu'elles retiennent), 
d'autre part à former de nouvelles boucles qui deviendront les mailles du rang 
suivant.

\newpage

\section{Présentation du logiciel}

Le nom TriComp de notre logiciel vient de la contraction des mots «\,tricot\,» et «\,compilateur\,», 
à juste titre puisqu'il s'agit d'un compilateur de tricot. En effet, le but du logiciel est de 
transformer une description globale d'un tricot en instructions à effectuer par le tricoteur 
pour le réaliser. Ces instructions sont écrites en langage naturel. Le logiciel se compose de trois parties :
\begin{itemize}
 \item \textbf{Un langage descriptif}, TriLang, qui permet à l'utilisateur de décrire globalement son ouvrage (la forme, la taille, les motifs, \dots)
 \item \textbf{Un compilateur} qui effectue la transformation de la description en une liste d'instructions
 \item \textbf{Une interface graphique} qui permet à la fois de visualiser le tricot décrit grâce à Trilang, 
   de faire appel au compilateur, et d'afficher la suite des instructions. \\
\end{itemize}

Le logiciel fonctionne de la manière suivante :
\begin{itemize}
 \item L'utilisateur décrit l'ouvrage qu'il veut effectuer grâce au langage descriptif. Il fait cela dans un nouveau fichier texte, auquel il donne l'extension \texttt{.tricot}, et qu'il range dans le dossier de son choix.
 \item L'utilisateur lance ensuite l'interface graphique. Une fois dans l'interface graphique il peut ouvrir son fichier \texttt{.tricot}. Il obtient une visualisation sous forme de patron de l'ouvrage qu'il a décrit précédemment.
 \item L'utilsateur peut, toujours dans l'interface graphique, utiliser quelques outils d'éditions qui ont été implémentés (modification du motif d'un ou plusieurs trapèze(s)). Il peut sauvegarder les modifications qu'il effectue.
 \item Une fois satisfait du modèle qu'il a défini, l'utilisateur peut, toujours dans l'interface graphique, faire appel au compilateur. Celui-ci va tout d'abord effectuer quelques vérifications pour s'assurer que l'ouvrage est tricottable (il va par exemple vérifier qu'il n'y a pas de cycle dans les pièces définies). Il va ensuite générer la liste des instructions à suivre afin de tricotter l'ouvrage. Ces intructions sont affichées dans l'interface graphique, et l'utisateur n'a plus qu'à les suivre pour obtenir son tricot.
\end{itemize}

La figure \ref{fct} montre schématiquement le fonctionnement décrit ci-dessus. % Si quelqu'un trouve mieux que montre...

\begin{figure}[!ht]
	\centering
	\begin{tikzpicture}[thick,node distance= 2.5cm]
	  \node[draw=none] (u) {Utilisateur};
	  \node[draw,rectangle, below of=u] (a) {Description du tricot};
	  \node[inner sep=0,minimum size=0,right of=a] (k) {}; % invisible node
	  \node[draw,rectangle,right of=k] (b) {Rendu visuel du patron};
	  \node[draw,rectangle,below of=a] (c) {Instructions pour tricoter};
	  \node [draw=none, below right of=k, node distance=1.75cm] {Compilateur};

	  % 1st pass: draw arrows
	  \draw[vecArrow] (u) -- node [right] {TriLang} ++ (a) ;
	  \draw[vecArrow] (a) -- node [above] {GUI} ++ (b) ;
	  \draw[vecArrow] (k) |-  (c) ;

	  % 2nd pass: copy all from 1st pass, and replace vecArrow with innerWhite
	  \draw[innerWhite] (a) to (b);
	  \draw[innerWhite] (k) |- (c);

	  % Note: If you have no branches, the 2nd pass is not needed
	\end{tikzpicture}
	\caption{Fonctionnement du logiciel}
	\label{fct}
\end{figure}

\section{Travail réalisé}

Le travail que nous prévoyions de réaliser impliquait d'une part de
définir différents langages de «\,programmation\,» du tricot (un langage
de haut niveau pour décrire de manière rigoureuse les ouvrages, et un
langage de bas niveau correspondant aux mailles à tricoter), d'autre
part d'écrire un compilateur réalisant la traduction d'un langage vers
l'autre, et enfin de réaliser une interface graphique permettant de
les manipuler intuitivement et de visualiser les pièces en cours de
définition. Nous détaillons dans cette partie le travail qui a été
réalisé dans chacun de ces objectifs. Nous rappelons que tout le
travail (logiciel, documents et site web) sont disponibles et
consultables sur le dépôt Git du projet
TriComp (\url{https://github.com/TriComp/}).

\subsection{Définition des langages}

Nous avons été amenés au cours de ce projet à définir deux langages,
l'un purement statique, puisque servant à décrire les ouvrages de
tricot, l'autre dyamique, car décrivant le processus de réalisation du
tricot. Nous les détaillons ici.

\subsubsection{Langage descriptif}

Une première version du langage descriptif avait été détaillée dans le
rapport mi-parcours. Quelques modifications ont été effectuées
depuis. Nous exposerons donc ici le langage final, les modifications
effectuées ainsi que leur intêret, le tout illustré par un exemple. \\


L'idée générale est de concevoir un langage capable décrire un tricot
de la façon la plus globale possible. Dans les (rares) logiciels de
tricot existants, le tricot est décomposé en sa plus petite entité
: la maille (cf figure \ref{logiciel}). L'avantage de cette
décomposition atomique est que l'on peut réaliser des choix de points
(voire de couleurs) avec une très bonne précision. 
Son gros inconvénient est qu'il est souvent
pénible de travailler avec cette représentation trop précise, car les tricots 
ont en général une taille assez importante. % TODO Une idée de la taille d'un tricot "classique"?
Notre démarche, quant à elle, propose une vision du tricot plus globale, 
et donc beaucoup plus simple à manipuler.

\begin{figure}[!ht]
  \centering \includegraphics[scale=0.3]{../img/grid.jpg}
  \caption{Le logiciel Stitch Designer propose une vue très précise du
    tricot, mais celui-ci a donc une taille beaucoup trop importante
    (comme en témoignent les barres de défilement à droite et en bas
    de la fenêtre)}
  \label{logiciel}
\end{figure}

L'élément de base dans ce langage est le trapèze. Un trapèze définit
une zone où l'on trouve un même motif. Un trapèze est défini par sa
hauteur, le décalage de sa base supérieure, et la longueur de ses
bases (voir figure \ref{trapeze}). Ce choix se justifie par le fait
que le trapèze est l'unité atomique la plus générale possible dans un
ouvrage. En effet, comme le tricot est réalisé rang après rang, la pièce
tricotée d'un tenant a deux côté parallèles. En revanche, grâce aux 
augmentations et aux diminutions les deux autres cotés ne sont pas 
forcément parallèles. Par l'assemblage de trapèzes, nous arrivons
à obtenir n'importe quelle pièce tricotable.

\begin{figure}[!ht]
  \centering \includegraphics[scale=0.5]{../img/trapeze.jpg}
  \caption{Les paramètres d'un trapèze que l'on utilise dans la
    description}
  \label{trapeze}
\end{figure}

Ainsi, un tricot est subdivisé en trapèzes qui sont reliés les uns les autres
d'une certaine manière : un trapèze connaît l'entité qui le suit
directement (dans le parcours du tricot) via un pointeur. Il se peut
que plusieurs trapèzes suivent un même trapèze : dans ce cas, on 
ajoute des aiguilles afin de pouvoir continuer à tricoter les deux branches 
indépendamment. Cette situation se modélise
dans le langage via le mot-clé \texttt{split}.  Inversement, il se
peut que plusieurs trapèzes aient le même successeur, c'est à ce
moment là que plusieurs morceaux peuvent être réunis afin de libérer
une ou plusieurs aiguilles : c'est un \texttt{link}.

Notre format de fichier nous autorise à définir différentes pièces
avec différents noms. Ces pièces permettent de décomposer une pièce plus importante,
notamment en cas de présence de \texttt{link}, qui prennent en entrée
le nom de la pièce suivante.

La figure \ref{langage} illustre et résume les possibilités du
langage, via une description d'un poncho.
\begin{figure}[!ht]
	\centering
	\begin{tikzpicture}[scale = 0.05]
		\tikzstyle{fleche}=[->,>=latex,line width=1mm]
                \fill[color=gray!20] (0,0) -- (60,0) -- (60,120) --
                (0,120) -- cycle; \fill[color=white] (30,30) --
                (45,60) -- (30,90) -- (15,60) -- cycle; \draw[thick]
                (0,0) -- (60,0) -- (60,120) -- (0,120) -- cycle;
                \draw[thick] (0,30) -- (60,30) ; \draw[thick] (0,90)
                -- (60,90) ; \draw[thick] (0,60) -- (15,60) ;
                \draw[thick] (45,60) -- (60,60) ; \draw[thick] (30,30)
                -- (45,60) -- (30,90) -- (15,60) -- cycle;
                \draw[fleche, color=red] (30,20) -- (15,40);
                \draw[fleche, color=red] (30,20) -- (45,40);
                \draw[color=red] (30,15) node{split}; \draw[fleche,
                  color=blue] (8,50) -- (8,70); \draw[fleche,
                  color=blue] (52,50) -- (52,70); \draw[color=blue]
                (30,60) node{next}; \draw[fleche,color=vert] (15,80)
                -- (30,100); \draw[fleche,color=vert] (45,80) --
                (30,100); \draw[color=vert] (30,105) node {link};
                \draw (30,-10) node{start}; \draw[fleche] (30,-5) --
                (30,10); \draw (30,130) node{stop}; \draw[fleche]
                (30,110) -- (30,125); \draw[-*,ultra thick] (75,30) --
                (45,15); \draw (90,35) node{\texttt{trapeze(\dots)}};
	\end{tikzpicture}
	\caption{Les possibilités du langage illustrées sur un tricot
          (en l'occurence un poncho). Une attention toute particulière
          est portée aux jonctions entre les différents trapèzes.}
	\label{langage}
\end{figure}

Comme nous le disions précédemment, quelques modifications ont été
apportées au niveau de ce langage depuis le rapport de
mi-parcours. Deux choses importantes ont été modifiées :
\begin{itemize}
	\item La largeur inférieure des trapèzes est cette fois-ci
          déduite de la largeur supérieure du trapèze précédent, et ne figure
          donc plus dans les paramètres du trapèze. En revanche, \texttt{start}
          et \texttt{split} prennent en argument la largeur inférieure du ou
          des trapèze(s) qui les suivent. Cette modification
          permet d'éviter des redondances sans se compliquer la tâche.
	\item Les \texttt{split} peuvent générer plus de deux pièces
          différentes. Par conséquent, les \texttt{link} ont
          également été modifiés : ils se lient avec leur successeur
          via le nom du successeur, suivi de la position sur ce
          successeur (alors qu'avant, un simple positionnement
          gauche/droite suffisait). Nous avons effectué cette
          modification car certains tricots (par exemple une salopette)
          ne pouvait être écrit de façon simple avec le précédent
          langage. Ceci est du au fait que précédemment, le
          positionnement des pièces après un \texttt{split} était déterministe
          (une pièce à gauche, une pièce à droite) et donc
          restrictif. Cette partie a apporté beaucoup plus de
          modifications dans le code que la modification précédente.
\end{itemize}

Au final nous obtenons un langage permettant de représenter une quantité importante 
de tricots d'un point de vue assez original par rapport à ce qui est fait dans les 
autres logiciels de tricot. Les tricots qui suivent notre langage portent l'extension 
\texttt{.tricot}.

À titre d'exemple, voici le code décrivant une petite écharpe faite uniquement avec du 
point mousse :\\

\texttt{Name : toy\_scarf}\\
\texttt{Description : "The easiest knit you can make"}\\
\texttt{piece my\_piece := start 20 }\\
\texttt{|| trapezoid (height : 60, shift : 0, upper\_width : 20, pattern : point\_mousse)}\\
\texttt{ || stop} \\

Un exemple plus complexe se trouve en annexe. De plus, la documentation complète du langage
est disponible sur le site de TriComp (\url{http://tricomp.github.io}).


\subsubsection{Langage de bas niveau}

Le \emph{langage de bas niveau} correspond aux instructions que doit suivre l'utilisateur 
pour pouvoir réaliser son tricot. Ces instructions sont produites par le compilateur, 
dont on détaillera le fonctionnement dans la prochaine partie. Ici nous présentons le 
format (ou langage) utilisé pour présenter ces instructions, proches de ce que l'on peut 
trouver dans un manuel de tricot.

Ce langage doit refléter exactement les mêmes informations que celles que l'on pourrait 
trouver dans un manuel de tricot : les points à tricoter. Dans les manuels, la périodicité 
des motifs est exploitée afin de ne pas donner une instruction pour chaque maille. Cette 
périodicité se traduit à la fois sur un rang (horizontalement, voir la figure \ref{instruction1}) 
et sur un ensemble de rangs (verticalement, voir la figure \ref{instruction2}).  

\begin{figure}[!ht]
	\centering
	\fbox{\begin{minipage}{0.9\textwidth}
	\begin{lstlisting}^^J
Ligne 1 : 140 fois 1 maille endroit, 1 maille envers^^J
	\end{lstlisting}
	\end{minipage}}
	\caption{Ici, on repète une succession de points plusieurs fois sur un même rang.}
	\label{instruction1}
\end{figure}

\begin{figure}[!ht]
	\centering
	\fbox{\begin{minipage}{0.9\textwidth}
	\begin{lstlisting}^^J
Répetez 245 fois le motif suivant :^^J
Ligne 1 : 140 fois 1 maille endroit, 1 maille envers^^J
Ligne 2 : 140 fois 1 maille envers, 1 maille endroit^^J
	\end{lstlisting}
	\end{minipage}}
	\caption{Ici, on repète une succession de rangs plusieurs fois (verticalement).}
	\label{instruction2}
\end{figure}

Les deux figures précédentes illustrent des instructions que nous pouvons obtenir avec notre logiciel. 
Dans un manuel de tricot, elles seraient volontairement moins verbeuses, à cause du nombre important 
de motifs différents à décrire que contiennent ces livres. \\

Mais tout tricot ne peut être construit uniquement à partir d'instructions de ce type. En effet, nous 
avons vu qu'avec les \texttt{split} et les \texttt{link} on peut avoir à ajouter (ou enlever) 
une ou plusieurs aiguilles, et cette information doit figurer parmi les instructions renvoyées.

Lors d'un \texttt{split}, le tricoteur laisse de côté une partie de son ouvrage sur une aiguille et prend une autre aiguille afin de tricoter une des branches du split. Les intructions alors données à l'utilsateur sont les suivantes :
\begin{itemize}
 \item On donne à l'utilisateur les instructions pour tricoter la premières branche après le split.
 \item Une fois cette branche terminée on demande à l'utilsateur, selon les cas, de fermer les mailles correspondant à cette branche ou de laisser son ouvrage sur l'aiguille.
 \item On donne les instructions pour la branche suivante.
 \item Et ainsi de suite.
\end{itemize}

Lors d'un \texttt{link}, le tricoteur doit assembler, c'est à dire tricoter à la suite avec le même fil, des partie d'ouvrages ayant précédemment été laissée de côté sur des aiguilles. L'instruction correspondante correspond tout simplement à dire à l'utilsateur d'``assembler les dépendances''. En effet, l'utilisateur ayant sous les yeux le patron du tricot qu'il réalise, il ne peut y avoir d'ambiguïté. \\

% TODO : Donnner exemple de code généré quand split et link ?

Enfin, la dernière information importante concerne le début et la fin du tricot. En effet, pour commencer un tricot, 
le tricoteur doit \emph{monter des mailles} (c'est-à-dire créer une première rangée de mailles sur son 
aiguille) tandis qu'à la fin il doit les fermer. 

Les premières mailles sont particulières, car elles ne 
reposent pas sur des mailles précédemment tricotées. Nous indiquons donc au début du tricot le nombre de 
mailles à monter.
Nous indiquons également les mailles à fermer quand besoin est.
Ces étapes ne figurent pas dans les manuels de tricot, car ceux-ci traitent souvent de 
motifs, et non de pièces. \\

% + Fermer les mailles (d'ailleurs on ferme bien les mailles lors d'un split avec trou, mais on ne les ferme pas à la fin !!)

Ainsi, les instructions que renvoie notre logiciel sont à la fois précises comme celles d'un manuel, mais 
également claires et faciles à suivre pour un être humain ayant des connaissances basiques en tricot. 

\subsection{Compilateur} % TODO : à détailler

Le rôle du compilateur est, à partir du fichier \texttt{.tricot}, d'une part de 
s'assurer qu'il décrit un tricot possible (par exemple qu'il ne contient
pas de cycle, que tous les \texttt{link} ont un sens, etc \dots), et 
d'autre part de générer des instructions  dans le langage de bas niveau.

Il est implémenté en OCaml, et il utilise la bibliothèque standard alternative
Core. L'analyse syntaxique utilise le générateur d'analyseur syntaxique Menhir,
et est robuste: il émet des messages d'erreurs précis lorsque le fichier d'entrée 
n'est pas valide.

Pour ce faire, il créé un graphe de dépendance entre les différentes pièces
du tricot: un arc dénote qu'il faut tricoter une pièce avant l'autre. On peut
alors utiliser des algorithmes de graphes de la littératures, par exemple pour 
la détection de cycle.

\subsubsection{Parcours du tricot}

Pour la génération des instructions bas niveau, on parcourt le graphe 
dans un ordre topologique qui est en plus adapté au tricot.

Le parcours du tricot utilise l'algorithme 1, qui est un algorithme de
dataflow à work list, très courant en compilation. Ici le workset est
implémenté à l'aide d'une pile, ce qui nous permet de parcourir
complètement une branche de tricot avant de s'occuper d'une
autre. Ainsi on ne demande pas au tricoteur de faire un va-et-vient
entre plusieurs aiguilles si ce n'est pas nécessaire.

\begin{algorithm}\label{algo}
\caption{\textsc{Algorithme de parcours du tricot}}
\begin{algorithmic}[1]
\State WorkSet $\leftarrow$ racines du graphe.
\State Done $\leftarrow$ ($\lambda n \to \varnothing$) // Un dictionnaire
\While{$WorkList \neq \varnothing$}
  \State x $\leftarrow$ Pop\,(WorkList)
  \State Imprimer les instructions pour x
  \For{y $\in$ Successors(x)}
    \State Done(y) $\leftarrow$ Done(y) $\cup$ \{ x \}
    \If{Done(y) $=$ Predecessors(y)} // On a parcouru toutes les dépendances de x
      \State WorkSet $\leftarrow$ WorkSet $\cup$ \{ y \}
    \EndIf
  \EndFor
\EndWhile
\end{algorithmic}
\end{algorithm}

\pagebreak


\subsection{Interface graphique}

Nous exposons dans cette partie le travail qui a été fourni concernant
l'élaboration de l'interface graphique, notamment les caractéristiques
de l'interface, les fonctionnalités ainsi que des détails
d'implémentation.

\subsubsection{Caractéristiques et fonctionnalités}

Avant toute chose, il nous fallait cibler les caractéstiques que
devait avoir l'interface. Destinée à être utilisée par des tricoteurs,
celle-ci devait être facile d'utilisation. L'utilisateur dispose des
outils indispensables à toute interface (ouverture, sauvegarde de
fichiers, zoom\dots) mais également d'outils propres au tricot et aux
objectifs de TriComp : un bouton pour générer les instructions, ainsi
que quelques outils d'édition de tricot (choix des points sur les
trapèzes).

Le choix du motif à mettre sur chaque trapèze peut être fait parmi une
dizaine de points disponibles via des boutons. Une amélioration future
pourrait être de permettre à l'utilisateur de définir ses propres points.

L'interface est divisée en trois parties, comme on peut le voir sur la 
figure \ref{interface} : une partie contenant les
outils d'édition de tricot, une fenêtre où est affiché le tricot, et
une zone où se trouve la liste des instructions générées par le
compilateur.

\begin{figure}[!ht]
  \centering \includegraphics[scale=0.35]{../img/interface.jpg}
  \caption{L'interface graphique de TriComp. En \textcolor{rouge}{rouge}, 
    la zone de choix des points, en \textcolor{vert}{vert}, la zone de 
    visualisation du tricot en cours, en \textcolor{bleu}{bleu}, la 
    zone où s'affichent les instructions.}
  \label{interface}
\end{figure}

% TODO : inclure des images 

\pagebreak

\subsubsection{Détails d'implémentation}

L'interface a été créée à l'aide de la bibliothèque Qt. Une grande
partie du projet a été intégrée au sein de la partie Qt, pour des
questions de simplicité au niveau de l'intégration. Seule la partie
compilation (écrite en OCaml) est dissociée de la partie Qt, et est
appelée dans l'interface via des appels systèmes.

Détailler l'ensemble des structures offertes par Qt utilisées dans le projet
serait ici long et inutile, le lecteur pourra se référer au code
disponible sur la page Github\footnote{Rappel :
  \url{https://github.com/TriComp/}}. On mentionnera juste que pour
l'affichage, on ne travaille pas directement sur le tricot en temps
qu'objet, mais sur une représentation du tricot formée
d'\texttt{items} (qui forme, en quelque sorte, une couche
supérieure). Même si les deux représentations sont isomorphes
(grossièrement un \texttt{item} pour un trapèze du tricot), cela
permet d'éditer autant qu'on le veut sans modifier l'objet de base
(sauf lors d'une sauvegarde).

\subsection{Communication}

Nous détaillons dans cette partie la partie communication du projet,
qui englobe la communication interne et externe (via le site web).

\subsubsection{Site Web}

Un site web a été déployé à l'adresse \url{http://tricomp.github.io}, il est hébergé
sur Github (utilisé comme serveur Git du projet). Le site est basé sur Jekyll, un 
CMS en Ruby spécialisé dans les blogs et qui a la particularité de ne pas utiliser 
de base de données, avec l'avantage que Github gère Jekyll automatiquement. Le site 
est notamment une vitrine pour le projet : il en contient une présentation rapide, 
avec des liens vers le code source sur Github, ainsi que différents articles autour du projet.
La page dispose également d'une page pour aiguiller rapidement l'internaute anglophone.

% TODO un onglet téléchargement
% TODO un onglet regrouppant ts les documents

\subsubsection{Communication au sein du projet}

La communication au sein du projet s'est principalement effectuée par le biais de notre liste
de diffusion (tricomp [at] ens-lyon.fr). L'échange des idées au début du projet s'est 
fait par l'intermédiaire d'un pad hébergé par l'association AliENS (\url{http://pad.aliens-lyon.fr/p/tricomp}). 
Nous nous réunissions également toutes les semaines avec notre encadrant afin de faire le point 
sur l'avancement du projet, de discuter de choix (tels que la syntaxe du langage descriptif, 
la modélisation à adopter pour les points, les fonctionnalités du logiciel final\dots), mais 
également au cours de séances non officielles où certains groupes se réunissaient 
afin de résoudre certaines problématiques un peu plus coriaces (notamment par rapport au langage, 
au compilateur ou à la gestion de l'éditeur de l'interface).

\section{Améliorations possibles}

Le manque de temps et de moyens humains nous a empêché de mettre en
place la totalité des fonctionnalités que nous avions prévues. Cependant, 
ce projet porte en lui un potentiel certain ; nous détaillons
ici quelques améliorations auxquelles nous avons songées, sachant qu'elles verront
peut-être le jour (certains membres de l'équipe étant intéressés par la 
poursuite du projet).

\subsection{Langage descriptif}

En ayant plus de temps à notre disposition, nous pourrions enrichir le langage de manière à permettre une personnalisation plus importante des motifs. 
Ceci pourrait se faire sous deux angles différents et complémentaires :
\begin{itemize}
 \item Une idée intéressante serait de laisser l'utilisateur définir ses propres points, en plus de ceux que nous avons prédéfinis. En effet, les possibilités de points sont infinies, et l'utilisateur pourrait ainsi exprimer toute sa créativité.
 \item Une autre idée serait de permettre à l'utilisateur de définir des zones dans son ouvrage où tel motif serait appliqué, au lieu de ne permettre le changement de motif que lors d'un changement de trapèze. En effet, notre langage tel quel ne permet pas par exemple de définir un pull dont la moitié gauche serait en point mousse et la moitié droite en jersey : la séparation entre les motifs se fait forcément en bandes horizontales. Bien qu'en soit pas très compliqué à mettre en oeuvre avec un repérage absolu, la définition des zones entraînerait des vérifications supplémentaires à effectuer auxquelles il faut bien penser.\\
\end{itemize}

De même, les tresses n'ont pas été intégrées au langage. Elles pourraient être intégrées dans un futur relativement proche en utilisant un paradigme similaire à celui utilisé pour décrire les points :
\begin{itemize}
 \item Il faudrait définir un format décrivant une tresse. Par exemple donner le nombre de brins de la tresse, la largeur de ces brins, et une description de la manière dont ces brins se croisent sous forme d'un motif minimal à répéter. De même que pour les points, l'idéal serait à terme que l'utilisateur puisse définir ses propres tresses.
 \item On pourrai se servir également de la définition de zones dans le tricot afin de décrire où positionner les tresses. \\
\end{itemize}

En outre, il pourrait être intéressant d'inclure dans le langage une description des coutures à effectuer afin d'assembler le tricot à la fin. En effet, bien que cette étape ne relève pas du tricot en tant que telle (et c'est pour cette raison qu'elle n'a pas été traitée jusqu'à présent), elle permettrait à long terme l'échange de modèles de tricots sous forme de fichier \texttt{.tricot} entre les tricotteurs.


\subsection{Compilateur}

Le compilateur actuel ne gère que les rectangles. La gestion des trapèzes 
(l'idée centrale du langage) serait un des premiers points sur lequel nous
pourrions nous pencher de nouveau. A travers cette question se pose notamment
les problèmes dela gestion des augmentations et des diminutions, notamment pour
des motifs complexes. Nous avons déjà réfléchi à un algorithme pour répartir ces 
augmentations et diminutions. Celui-ci se trouve dans la section 4.

\subsection{Interface graphique et édition}

Du côté de l'interface, une grande partie du travail que nous n'avons
pas eu le temps d'aborder portait sur l'édition des tricots. 
En effet pour l'instant, le seul outil d'édition qui a été implémenté est celui qui permet de changer le motif des trapèzes. Nous voudrions étoffer ces outils d'éditions, permettant à l'utilisateur par exemple de chager la taille des pièces de son tricot. L'idée serait qu'à terme l'utilsateur puisse entièrement décrire son tricot dans l'interface graphique sans avoir à passer par le langage que nous avons défini (qui peut, bien qu'il soit facile de prise en main, rebuter certains des utilisateurs).

Par ailleurs d'autres améliorations plus mineures (afficher le nom des pièces sur le patron pour faciliter la lecture des instructions par exemple) peuvent être envisagées.

% je mets ça là, mais je ne sais pas à quel point c'est pertinent : ne pas hésiter à 
% en discuter sur la ML
\subsection{Interfaçage avec des machines à tricoter}

L'intégration de machines à tricoter dans notre projet avait été écartée dès le 
début du projet, d'une part parce qu'aucun membre de l'équipe n'avait les compétences 
en électronique requises pour un tel travail, d'autre part parce que nous savions que 
le temps nous manquerait pour nous en occuper.

Cependant, cela semble un prolongement raisonnable de notre projet, dans le sens où 
cela pousserait l'automatisation du tricot jusque dans sa réalisation. Il nous faudrait 
pour cela tout d'abord nous associer avec des personnes ayant de l'expérience dans le 
domaine des machines à tricoter, puis travailler à adapter les instructions bas niveau 
à l'utilisation par une machine à tricoter (une différence majeure entre un tricoteur 
humain et une machine à tricoter étant que la machine à tricoter tricote toutes les 
mailles d'un rang en parallèle, tandis que le tricoteur les tricotera de manière séquentielle).

\section{Résultats théoriques}

\subsection{Algorithmes de répartition des diminutions}

\paragraph{Rappels}

Une \emph{augmentation} est une opération consistant à ajouter une maille au rang en cours : ainsi, à la suite d'une 
augmentation, ce rang comportera une maille de plus que le rang précédent, la largeur du tricot aura donc augmenté.

Une \emph{diminution} est l'opération inverse d'une augmentation : on supprime une maille au rang en cours, et on fait 
donc diminuer la largeur du tricot.

% ici schéma d'un jeté et de deux mailles tricotées ensemble
% -> préciser que c'est *une* manière parmis d'autres de faire des augmentations / diminutions

\paragraph{Réalisation des trapèzes}

Dans l'optique d'intégrer des trapèzes non rectangulaires au langage, nous nous sommes penchés sur la manière de répartir des 
diminutions et augmentations au cours des rangs, de manière à ce que les côtés apparaissent rectilignes.
% Peut-être plus détailler ce que sont les augmentations/diminutions (schéma ?)
Nous nous sommes donc penchés du côté des algorithmes d'image, permettant de tracer une ligne apparaissant la plus droite 
possible en utilisant des pixels discrets.

Cet élément du langage n'a pas encore été implémenté ; cependant, en prévision de l'intégration des diminutions et augmentations 
au langage, nous avons fait quelques recherches, et avons décidé d'utiliser l'algorithme de \textsc{Bresenham}
\footnote{\url{http://fr.wikipedia.org/wiki/Algorithme_de_trac\%C3\%A9_de_segment_de_Bresenham}}.

\subsection{Allocation d'aiguilles}

Lors de la génération des instructions, on souhaiterait déterminer le nombre d'aiguilles nécessaires à la réalisation du tricot, 
et si possible le minimiser. Ce problème ressemble fort à celui de l'allocation de registres en compilation, et on peut montrer 
qu'il lui est en fait équivalent.

En effet, faisons l'analogie suivante : on fait correspondre les aiguilles supplémentaires (par rapport aux 2 aiguilles de base) 
aux temporaires dans lesquels stocker les variables, et les variables aux zones où le tricot est séparé en deux, introduites par 
les \texttt{split} (en gardant en tête qu'on peut évidemment effectuer un \texttt{split} sur une branche déjà issue d'un 
\texttt{split}). Ainsi, on peut réaliser la même réduction de ce problème depuis la coloration de graphe qu'à celui de l'allocation 
de temporaires.

Cela nous permet de montrer que le problème de l'allocation d'aiguilles est \textsc{NP}-complet ; il est donc fort probable 
que nous devions utiliser des heuristiques pour déterminer le nombre d'aiguilles à prévoir pour réaliser un tricot. En pratique,
ce nombre est rarement significativement élevé, il est donc probable que ces heuristiques soient suffisants.

\section*{Conclusion}

Au terme des quatre mois qui nous étaient donnés pour mettre en place le logiciel, nous avons créé un langage robuste, ainsi qu'un 
logiciel fonctionnel sur ses différentes parties (compilateur, interface), ce dernier étant cependant incomplet par rapport aux 
objectifs que nous nous étions fixés en début de projet. La partie implémentée répond au besoin primaire que nous avions ciblé : 
la définition d'un langage de description de tricots, l'édition (au moins partielle) de ces tricots et 
la génération d'instructions associées, le tout via une interface graphique.

\newpage

\appendix

\section{Exemples de codes sources}

\subsection{Code \texttt{.tricot}}

\lstinputlisting{ponzo.tricot}


\subsection{Instructions en sortie du compilateur}

\setlength{\parindent}{0cm}

Instructions pour "ponzo":\\

Début de la pièce "my\_piece":\\
Montez 840 mailles.\\

Répetez 100 fois le motif suivant :\\
Ligne 1 : 105 fois 5 mailles envers, 3 mailles endroit\\
Ligne 2 : 105 fois 3 mailles endroit, 5 mailles envers\\


Répetez 420 fois le motif suivant :\\
Ligne 1 : 420 fois 1 maille endroit, 1 maille envers\\
Ligne 2 : 420 fois 1 maille endroit, 1 maille envers\\


Répetez 245 fois le motif suivant :\\
Ligne 1 : 35 fois 5 mailles envers, 3 mailles endroit\\
Ligne 2 : 35 fois 3 mailles endroit, 5 mailles envers\\

Laissez l'aiguille de côté pour le moment.\\
Fermez 280 mailles\\

Répetez 245 fois le motif suivant :\\
Ligne 1 : 35 fois 5 mailles envers, 3 mailles endroit\\
Ligne 2 : 35 fois 3 mailles endroit, 5 mailles envers\\


Début de la pièce "top":\\
Assemblez les dépendances.\\

Répetez 420 fois le motif suivant :\\
Ligne 1 : 420 fois 1 maille endroit, 1 maille envers\\
Ligne 2 : 420 fois 1 maille endroit, 1 maille envers\\


Répetez 105 fois le motif suivant :\\
Ligne 1 : 105 fois 5 mailles envers, 3 mailles endroit\\
Ligne 2 : 105 fois 3 mailles endroit, 5 mailles envers\\


Félicitations, vous avez fini!


\end{document}
