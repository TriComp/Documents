\documentclass{article}

\usepackage[top=3cm, bottom=3cm, left=3cm, right=3cm]{geometry}

\usepackage[french]{babel}
\usepackage[utf8]{inputenc}
\usepackage[T1]{fontenc}

\usepackage[a4paper,colorlinks,linkcolor=darkgray,citecolor=red,urlcolor=blue]{hyperref}
\usepackage{pdfpages}
\usepackage{graphicx}
\usepackage{caption}
\usepackage{amsthm}
\usepackage{listings}

\newtheorem{ex}{Exemple}

\title{Rapport final : TriComp}

\author{Équipe TriComp}

\date{}


\begin{document}

\maketitle

\newpage

\tableofcontents

\newpage

\section*{Introduction}

Le tricot est un art qu'on imagine souvent réservé aux personnes âgées

Les magazines de tricot présententent souvent des modèles à réaliser soi-même, combinant différents points. Il est cependant vite fastidieux d'écrire la suite des 
instructions à suivre pour réaliser une pièce.
% ici, on explique nos motivations, la genèse du projet
% motivations : le tricot c'est bien, mais vite fastidieux de construire des modèles à la main
% Magazines de tricot : proposent souvent des modèles combinant de nombreux points différents, mais fastidieux de construire le modèle et de calculer où il faut faire quelle opération (tricoter une maille à l'endroit ou à l'envers, croiser des mailles, en ajouter, en enlever...)
% autre motivation : intérêt théorique, potentiels problèmes mathématiques intéressants peuvent surgir (en pratique on n'est pas allés assez loin pour les rencontrer)

\subsection*{Lexique du tricot}

% définition d'un certain nombre de mots : maille, point, ouvrage ; et description rudimentaire des techniques

\section{Travail réalisé}

\subsection{Définition des langages}

\subsubsection{Langage descriptif}

% donner un pointeur sur la documentation ici

\subsubsection{Langage de bas niveau}




\subsection{Compilateur}

\subsubsection{Langage descriptif vers langage bas niveau}

\subsubsection{Langage bas niveau vers langage naturel}




\subsection{Interface graphique}

% inclure des captures d'écran etc

% le midterm report met ici le site web, mais je ne suis pas si sûre que ça soit très pertinent
% -> autre section ? ("Communication")



\section{Améliorations possibles}

% ici, on parlera des autres versions du langage, des améliorations qu'on aurait pu apporter à l'interface graphique...

\subsection{Langage de description}

\subsection{Interface graphique}




\section{Résultats théoriques}

\subsection{Algorithmes de répartition des diminutions}

\subsection{Allocation d'aiguilles}





\end{document}
