\documentclass{article}

\usepackage[top=3cm, bottom=3cm, left=3cm, right=3cm]{geometry}

\usepackage[french]{babel}
\usepackage[utf8]{inputenc}
\usepackage[T1]{fontenc}

\usepackage[a4paper,colorlinks,linkcolor=darkgray,citecolor=red,urlcolor=blue]{hyperref}
\usepackage{pdfpages}
\usepackage{graphicx}
\usepackage{caption}
\usepackage{amsthm}
\usepackage{listings}

\newtheorem{ex}{Exemple}

\title{Rapport final : TriComp}

\author{}

\date{Vendredi 19 Décembre 2014}

\begin{document}

\makeatletter % Pour utiliser le "at" comme une commande interne.
  \begin{titlepage}
    \begin{center}
       {\LARGE \@title} \\
       \vspace{2cm}
       {\large \@date}
       \vspace{3cm}
    \end{center}
       {\large
       {William \textsc{Aufort} \hfill Julien \textsc{Bensmail} \\}
    \vspace{1cm}
       {\hfill coordinateur \\}
       {Agathe \textsc{Herrou}  \hfill Romain \textsc{Labolle} \\}
       \vspace{1cm}
       {chef de projet \\}
       \vspace{1.5cm}
       {Frédéric \textsc{Lang} \hfill Maxime \textsc{Lesourd} \\}
       {Laureline \textsc{Pinault} \hfill Léo \textsc{Stéfanesco} \\}}
       \vspace{2.5cm}
    \begin{abstract}
	Ce document présente le rapport final de notre projet TriComp. 
	% TODO : résumer le contenu du document
    \end{abstract}
  \end{titlepage}
\makeatother

\newpage

\tableofcontents

\newpage

\section*{Introduction}

Le tricot est un art que l'on imagine souvent réservé aux personnes âgées.

Les magazines de tricot présententent souvent des modèles à réaliser soi-même, combinant différents points. Il est cependant vite fastidieux d'écrire la suite des 
instructions à suivre pour réaliser une pièce. Notre but ici était donc de réaliser un logiciel permettant d'une part de décrire des pièces de tricot de manière \emph{user-friendly}, et, à partir de cette description, de générer une suite d'instructions à destination du tricoteur permettant de réaliser la pièce en question. 

Cette opération de traduction d'un langage de ``haut niveau'' (l'ensemble des pièces du tricot) vers un langage de ``bas niveau'' (les instructions) rappelait fortement le principe d'un compilateur, c'est pourquoi 
nous avons d'abord considéré notre projet comme un ``compilateur de tricot''. À l'instar de la compilation d'un langage de programmation informatique, cette 
compilation de tricot était susceptible de mener à des problèmes théoriques intéressants, à mesure que l'on enrichirait le langage.
% ici, on explique nos motivations, la genèse du projet
% motivations : le tricot c'est bien, mais vite fastidieux de construire des modèles à la main
% Magazines de tricot : proposent souvent des modèles combinant de nombreux points différents, mais fastidieux de construire le modèle et de calculer où il faut faire quelle opération (tricoter une maille à l'endroit ou à l'envers, croiser des mailles, en ajouter, en enlever...)
% autre motivation : intérêt théorique, potentiels problèmes mathématiques intéressants peuvent surgir (en pratique on n'est pas allés assez loin pour les rencontrer)

\subsection*{Lexique du tricot} % ou : Quelques mots sur le tricot

% définition d'un certain nombre de mots : maille, point, ouvrage ; et description rudimentaire des techniques

Le vocabulaire du tricot peut sembler un peu obscur aux non-initiés, voici donc quelques définitions :

\begin{itemize}
% + rang
\item une \emph{maille} est l'unité de base d'un tricot. Il s'agit de la boucle qui constitue l'étoffe en étant reliée à ses voisines, horizontalement car 
  partageant la même portion de fil, verticalement car ces boucles sont imbriquées les unes dans les autres. Elle peut être à l'endroit ou à l'envers, selon la 
  manière dont la boucle est réalisée.
\item un \emph{point} est une manière de combiner les différentes opérations réalisables sur les mailles (les tricoter à l'endroit, à l'envers, en tricoter 
  plusieurs ensemble, croiser plusieurs mailles \dots)
\item l'\emph{ouvrage} désigne la pièce de tricot tout entière, par exemple l'écharpe ou le pull.
\item une \emph{diminution} est la suppression d'une maille dans le cours de l'ouvrage ; une technique courante consiste à tricoter ensemble 
  deux mailles. Symétriquement, une \emph{augmentation} est une maille ajoutée dans le cours de l'ouvrage ; une technique 
  courante consiste à enrouler le fil sur l'aiguille pour former une nouvelle maille (on parle alors de \emph{jeté}).
\end{itemize}

Le tricot est réalisé à l'aide d'aiguilles, qui servent d'une part à maintenir les boucles libres (pour éviter qu'elles ne se défassent et libèrent les boucles du 
rang précédent qu'elles retiennent), d'autre part à former de nouvelles boucles.

\newpage

\section{Travail réalisé}

Le travail que nous prévoyions de réaliser impliquait d'une part de définir différents langages de ``programmation'' du tricot (un langage de haut niveau pour 
décrire de manière rigoureuse les ouvrages, et un langage de bas niveau correspondant aux mailles à tricoter), d'autre part d'écrire un compilateur réalisant la 
traduction d'un langage vers l'autre, et enfin de réaliser une interface graphique permettant de les manipuler intuitivement et de visualiser les pièces en cours de 
définition.
Nous détaillons dans cette partie le travail qui a été réalisé dans chacun de ces objectifs. Nous rappelons que tout le travail (logiciel, documents et site web) sont disponibles et consultables sur le dépôt Git du projet TriComp. (\url{https://github.com/TriComp/}).

\subsection{Définition des langages}

Nous avons été amenés au cours de ce projet à définir deux langages, l'un purement statique, puisque servant à décrire les ouvrages de tricot, l'autre dyamique, car 
décrivant le processus de réalisation du tricot.

\subsubsection{Langage descriptif}

Une première version du langage descriptif avait été détaillée dans le rapport mi-parcours. Quelques modifications ont été effectuées depuis. Nous exposerons donc ici le langage final, les modifications effectuées ainsi que leur intêret, le tout illustré par un exemple. \\

L'idée générale est de concevoir une langage capable décrire un tricot de la façon la plus globale possible. Dans les (rares) logiciels de tricot qui existent, le tricot est décomposé en sa plus petite entité : la maille (cf figure \ref{logiciel}). L'avantage de cette décomposition atomique est que l'on peut réaliser des choix de points avec une très bonne précision. Notre démarche, quand à elle, propose une vision du tricot plus globale, et donc beaucoup plus simple.

\begin{figure}[!ht]
  \centering
  \includegraphics[scale=0.3]{../img/grid.jpg}
  \caption{Le logiciel Stitch Designer propose une vue très précise du tricot, mais celui-ci a donc une taille beaucoup trop importante (comme en témoignent les barres de défilement à droite et en bas de la fenêtre)}
  \label{logiciel}
\end{figure}

L'élément de base dans ce langage est le trapèze. Un trapèze définit une zone où l'on trouve un même motif. Un trapèze est défini par sa hauteur, le décalage de sa base supérieure, et la longueur de ses bases (voir figure \ref{trapeze}).

\begin{figure}[!ht]
  \centering
  \includegraphics[scale=0.5]{../img/trapeze.jpg}
  \caption{Les paramètres d'un trapèze que l'on utilise dans la description}
  \label{trapeze}
\end{figure}

Un tricot est subdivisé en trapèzes qui sont reliés les uns les autres d'une certaine manière : un trapèze connaît l'entité qui le suit directement (dans le parcours du tricot) via un pointeur. Il se peut que plusieurs trapèzes suivent un même trapèze : c'est dans cette situation que l'on ajoute des aiguilles. Cette situation se modélise dans le langage via le mot-clé "\texttt{split}".
Inversement, il se peut que plusieurs trapèzes aient le même successeur, c'est à ce moment là que plusieurs "morceaux" peuvent être réunis afin de libérer une ou plusieurs aiguilles : c'est un "\texttt{link}".

Notre format de fichier nous autorise à définir différentes pièces avec différents noms. Ces pièces permettent de décomposer une pièce, notamment en cas de présence de \texttt{link}, qui prennent en entrée le nom de la pièce suivante. 

% TODO : Pour démarrer le tricot, comment fait-on ? Il avait été question à un moment de tri topologique. Ajouter une pièce fantôme (un split n-aire de départ) fonctionne si les différentes pièces de départ sont à la même hauteur

Comme nous le disions précédemment, quelques modifications ont été apportées au niveau de ce langage depuis le rapport de mi-parcours. Deux choses importantes ont été modifiées :
\begin{itemize}
	\item La largeur inférieure des trapèzes est cette fois-ci déduite de la largeur de la pièce précédente, et ne figure donc plus dans les paramètres du trapèze. Cette modification permet d'éviter des redondances sans se compliquer la tâche, notamment dans la partie de l'interface graphique.
	\item Les \texttt{split} peuvent générer plus de deux pièces différentes. Par conséquent, les \texttt{links} ont également été modifiés : ils se lient avec leur sucesseur via le nom du sucesseur, suivi de la position sur ce sucesseur (alors qu'avant, un simple  positionnement gauche/droite suffisait). Nous avons effectué cette modification car certains tricots (par exemple la salopette) ne pouvait être écrit de façon "simple" avec le précédent langage. Ceci est du au fait que précédemment, le positionnement des pièces après un split était déterministe (une pièce à gauche, une pièce à droite) et donc restrictif. Cette partie a apporté beaucoup plus de modifications dans le code que la modification précédente.
\end{itemize}

% TODO : reprendre un exemple + une figure (de préférence fléchée avec les split/link/next) pour illustrer

\subsubsection{Langage de bas niveau}

\subsection{Compilateur}

\subsubsection{Langage descriptif vers langage bas niveau}

\subsubsection{Langage bas niveau vers langage naturel}

\subsection{Interface graphique}

Nous exposons dans cette partie le travail qui a été fourni concernant l'élaboration de l'interface graphique, notamment les caractéristiques de l'interface, les fonctionnalités ainsi que des détails d'implémentation.

\subsubsection{Caractéristiques et fonctionnalités}

Avant toute chose, il nous fallait cibler les caractéstiques que devait avoir l'interface. Destinée à être utilisée par des tricoteurs, celle-ci devait être facile d'utilisation. L'utilisateur dispose des outils indispensables à toute interface (ouverture, sauvegarde de fichiers, zoom...) mais également d'outils propres au tricot et aux objectifs de TriComp : un bouton pour générer les instructions, ainsi que quelques outils d'édition de tricot (choix des points sur les trapèzes).

L'interface est divisée en trois parties : une partie contenant les outils d'édition de tricot, une fenêtre où est affiché le tricot, et une zone où se trouve la liste des instructions générées par le compilateur.

% TODO : inclure des images 

% inclure des captures d'écran etc

% le midterm report met ici le site web, mais je ne suis pas si sûre que ça soit très pertinent
% -> autre section ? ("Communication")



\section{Améliorations possibles}

% ici, on parlera des autres versions du langage, des améliorations qu'on aurait pu apporter à l'interface graphique...
Le manque de temps et de moyens humains nous a empêché de mettre en place la totalité des fonctionnalités que nous avions prévues, notamment en ce qui concerne les 
types de points gérés par le langage (la version actuelle ne gère que les points à base de mailles à l'endroit et à l'envers).

\subsection{Langage de description}
% à réécrire pour le rendre sexy et vendeur
En ayant plus de temps à notre disposition, nous pourrions intégrer au langage d'une part les diminutions et augmentations, d'autre part les tresses. Ces 
différentes 

\subsection{Interface graphique}




\section{Résultats théoriques}

\subsection{Algorithmes de répartition des diminutions}

\subsection{Allocation d'aiguilles}





\end{document}
