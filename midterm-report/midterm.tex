\documentclass{article}

\usepackage[top=3cm, bottom=3cm, left=3cm, right=3cm]{geometry}

\usepackage[french]{babel}
\usepackage[utf8]{inputenc}
\usepackage[T1]{fontenc}

\usepackage[a4paper,colorlinks,linkcolor=darkgray,citecolor=red,urlcolor=blue]{hyperref}
\usepackage{pdfpages}
\usepackage{graphicx}
\usepackage{caption}
\usepackage{tikz}
\usetikzlibrary{positioning,shapes,arrows,automata}
\tikzset{>=stealth'}

\title{Rapport d'activité mi-parcours : TriComp}

\author{}

\date{4 Novembre 2014}

\begin{document}

\makeatletter % Pour utiliser le "at" comme une commande interne.
  \begin{titlepage}
    \begin{center}
       {\LARGE \@title} \\
       \vspace{2cm}
       {\large \@date}
       \vspace{3cm}
    \end{center}
       {\large
       {William \textsc{Aufort} \hfill Julien \textsc{Bensmail} \\}
    \vspace{1cm}
       {\hfill coordinateur \\}
       {Agathe \textsc{Herrou}  \hfill Romain \textsc{Labolle} \\}
       \vspace{1cm}
       {chef de projet \\}
       \vspace{1.5cm}
       {Frédéric \textsc{Lang} \hfill Maxime \textsc{Lesourd} \\}
       {Laureline \textsc{Pinault} \hfill Léo \textsc{Stéfanesco} \\}}
       \vspace{2.5cm}
  \end{titlepage}
\makeatother


\newpage

\tableofcontents

\newpage

% Préparer le plan : 
%
%
%
%
%
%
%

\end{document}
