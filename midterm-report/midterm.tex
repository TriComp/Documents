\documentclass{article}

\usepackage[top=3cm, bottom=3cm, left=3cm, right=3cm]{geometry}

\usepackage[french]{babel}
\usepackage[utf8]{inputenc}
\usepackage[T1]{fontenc}

\usepackage[a4paper,colorlinks,linkcolor=darkgray,citecolor=red,urlcolor=blue]{hyperref}
\usepackage{pdfpages}
\usepackage{graphicx}
\usepackage{caption}
\usepackage{tikz}
\usetikzlibrary{positioning,shapes,arrows,automata}
\tikzset{>=stealth'}

\title{Rapport d'activité mi-parcours : TriComp}

\author{}

\date{4 Novembre 2014}

\begin{document}

\makeatletter % Pour utiliser le "at" comme une commande interne.
  \begin{titlepage}
    \begin{center}
       {\LARGE \@title} \\
       \vspace{2cm}
       {\large \@date}
       \vspace{3cm}
    \end{center}
       {\large
       {William \textsc{Aufort} \hfill Julien \textsc{Bensmail} \\}
    \vspace{1cm}
       {\hfill coordinateur \\}
       {Agathe \textsc{Herrou}  \hfill Romain \textsc{Labolle} \\}
       \vspace{1cm}
       {chef de projet \\}
       \vspace{1.5cm}
       {Frédéric \textsc{Lang} \hfill Maxime \textsc{Lesourd} \\}
       {Laureline \textsc{Pinault} \hfill Léo \textsc{Stéfanesco} \\}}
       \vspace{2.5cm}
    \begin{abstract}
	Ce document présente le rapport d'activité mi-parcours de notre projet TriComp. Y sont détaillés le travail fourni jusqu'à présent 
dans les différents groupes de travail ainsi que les divers modifications qui ont été effectuées par rapport à la proposition de projet.
    \end{abstract}
  \end{titlepage}
\makeatother


\newpage

\tableofcontents

\newpage

% Plan (proposition) : 
%  I] Travail fourni
%     a) Définition des différents langages
%     b) Interface graphique et site web
%     c) Divers(organisation,...)
%  II] Changements par rapport à la proposition 
%     a) Calendrier
%     b) Précisions diverses

\section{Travail fourni}

Nous exposons dans cette partie le travail fourni jusqu'à présent dans les différents groupes de travail.

\subsection{Définition des différents langages}

L'étape crutiale de définition des langages a été effectuée. Nous exposons ici avec précision ces différents langages.

\subsubsection{Instructions utilisateurs}

\subsubsection{Langage descriptif}

\subsection{Interface graphique}

Nous détaillons dans cette section les fonctionnalités implémentées jusqu'à présent au niveau de l'interface graphique.

\subsubsection{Squelette de l'interface / Aspect}

La première étape du développement de l'interface a été la mise en place de son squelette. Plus précisement, nous avons défini l'aspect 
général de l'interface ainsi que les différents outils que nous souhaitions mettre à disposition. Ces outils sont représentés à l'aide de 
boutons ou d'options encore inactives pour la plupart. On distingue notamment :
\begin{itemize}
  \item Les options relatives au logiciel TriComp (choisir un point, faire une tresse...) Ceux-ci peuvent être selectionnés grâce à des 
boutons dont l'aspect n'est pas encore fixé. % TODO : expliquer pourquoi (attente avancement de la partie affichage)
  \item Les options que l'on trouve dans tout logiciel (ouvrir, sauvegarder, quitter, ...). Ces options sont fonctionnelles (ou bientôt 
fonctionnelles) car les formats de données pour la sauvegarde ont été définis.
\end{itemize}

\subsection{Affichage des tricots}

\subsection{Site Web}

\section{Changement par rapport à la proposition}

\subsection{Calendrier}

\subsubsection{Représentation intermédiaire}

Il avait initialement été prévu de développer une représentation intermédiaire, où le tissu serait représenté sous la forme d'un graphe (les sommets 
correspondant aux mailles et les arêtes à la manière dont ils interagissaient), qui avait pour but primaire d'aider à la détection de configurations 
impossibles. Cependant, après avoir approfondi les langages de bas et haut niveau, nous nous sommes rendus compte que ce formalisme n'était pas 
intrinsèquement nécessaire, nous avons donc abandonné l'utilisation globale de cette formalisation, pour ne se concentrer que sur l'étude de motifs 
impossible ponctuelle et non systématique.

\subsection{Précisions diverses}

\section*{Conclusion}


\end{document}
